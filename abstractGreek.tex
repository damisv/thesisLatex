\thispagestyle{plain}
\begin{center}
    \Large
    \textbf{\e{iPM}}
    
    \vspace{0.4cm}
    \Large
    \e{Project Management Web Application}
    
    % \vspace{0.4cm}
    % \textbf{Author Name}
    
    \vspace{0.9cm}
    \huge
    \textbf{Περίληψη}
\end{center}
\pSpaceΗ παρούσα πτυχιακή εργασία πρόκειται για την κατασκεύη μιας διαδικτυακής εφαρμογής για την διαχείριση έργων.\\
\pSpaceΚύριως στόχος της εφαρμογής είναι η απλότητα χρήσης και ικανοποίηση βασικών αναγκών για την εύκολη και αποτελεσματική διαχείρηση ενός έργου.\\
\pSpaceΕπομένως, ο χρήστης έχοντας δημιουργήσει έναν λογαριασμό θα μπορεί να διαχειριστεί ή να ενταχθεί σε πολλαπλά έργα.Ένα έργο περιλαμβάνει ένα η περισσότερους χρήστες με διάφορους ρόλους (\e{manager, member, project manager}).Οποιοδήποτε μέλος της ομάδας θα μπορεί να ορίσει εργασίες (\e{assignments}), σε έναν η περισσότεροι μέλοι, χρησιμοποιώντας η μη της εξαρτήσεις εργασιών (\e{dependencies}).\\
\pSpaceΕπιπλέον, ο χρήστης θα έχει την δυνατότητα να παρακολουθεί όλες τις \e{assignments} που του έχουν ανατεθεί σε όλα τα ενεργά \e{project}, αλλά και σε κάθε έργο ιδιαιτέρως.Ο οπτικός τρόπος αναπαράστασης είναι διαθέσιμο και στις δύο περίπτωσεις σε μορφή πίτας και διάγραμμα \e{Gantt}.\\
\pSpaceΕπιπροσθέτως, κάθε έργο έχει ενα κανάλι επικοινωνίας για την ομαλή διεξαγωγή του έργου και ένα γενικό ημερολόγιο όπου προσθέτονται σύμβαντα προσωπικά η επί το έργο.\\
\pSpaceΔεύτερος στόχος της εργασίας αυτής ως προς την υλοποίησης της εφαρμογής, είναι να συνδιαστούν σωστά δύο γνωστά πρότυπα προγραμματισμού, δηλαδή "Αντικειμενοστραφής" και "Αντιδραστικό".\\
\pSpaceΟι τεχνολογίες που θα χρησιμοποιηθούν ακολουθούν το πακέτο "\e{MEAN}": \e{MongoDB} ως βάση δεδομένων, \e{ExpressJS} για την διοργάνωη \e{server-side}, το ισχυρό \e{Angular (6.0.0-beta7)} για \e{client-side} και \e{NodeJS} ως \e{server}, στα οποία προσθέτωνται το \e{Socket.io} και \e{RxJS}.