\subsection*{Συμπεράσματα}
\pSpaceΟι εφαρμογές και το διαδίκτυο αποτελούν σήμερα τα απαραίτητα εργαλεία στην διαχείριση έργων.Καθώς η τεχνολογία προχωράει και οι θεωρίες πολλαλασιάζονται περί του θέματος, οι χρήστες ψάχνουν την καλύτερη δυνατή λύση.Όμως, ο συνδυασμός των βασικών αναγκών, η εύκολη χρήση και η εξοικονόμιση χρόνου οδηγεί μια λύση που τον εκπληρώνει αυτόν τον συνδυασμό, ανάμεσα στις καλύτερες λύσεις που προσφέρονται στην αγορά.\\
\pSpaceΣτην αρχή της παρούσας εργασίας ο σκοπός που τέθηκε ήταν η κατασκεύη μια διαδικτυακής εφαρμογής για την διαχείριση έργων, που στοχεύει την απλότητα και εύκολη χρήση της.\\
\pSpaceΎστερα από κάποιες δοκιμές απο άτομα με διαφορετικό επίπεδο γνώσεων ως πρός το \e{Project Management}, συμπεράθηκε πως η εφαρμογή εκπληρεί τον σκοπό της.Ενώ στερεί σε αριθμό λειτουργιών, παρατηρήθηκε πως ο χρήστης ήταν ικανοποιημένος κυρίως από την απλή χρήση και την εξοικονόμηση χρόνο που αποδεικνυεί.Όπως ήταν αναμενόμενο, στις περιπτώσεις όπου τα έργα είναι μεγαλύτερου μεγέθους, απαιτείται μια λύση που να καλύπτει περισσότερες λειτουργίες.\\
\pSpaceΈνα δεύτερο σκοπό που έχει τεθεί, ως προς την υλοποίηση, ήταν ο σωστός συνδιασμός των δύο γνωστών πρότυπων προγραμματισμού, Αντικειμενοστραφής και Αντιδραστικό.\\
\pSpaceΤο δυσκολότερο κομμάτι για την επίτευξη αυτού του σκοπού, ήταν η συνήθεια στην Αντιδραστική σκέψη.Χρησιμοποιώντας αύτον τον πρότυπο, άλλες πρακτικές του προγραμματισμού έπρεπε να απορριφθούν καθώς ήταν ανούσιες η έκαναν τον κώδικα πολύπλοκο χωρίς κάποιο κέρδος.\\
\pSpaceΣυμπεράθηκε όμως, ότι ο συνδυασμός των δύο δεν συνιστάται αν οι προγραμματιστές δεν αποφασίσουν ομαδικά να ακολουθούν ένα αντιδραστικό τρόπο ανάπτυξης.Επίσης, η επίτευξη αυτού του σκοπού θεωρήθηκε ομαδικώς, ότι έχει ολοκληρωθεί κατά 85 $\%$.\\

\subsection*{Επεκτασιμότητα και βελτιώσεις}
\pSpaceΗ εφαρμογή βρίσκεται αύτην την στιγμή σε ένα πολύ βασικό στάδιο, αλλά μπορεί να αναπτυχθεί σε αρκετά υψηλότερο επίπεδο κρατώντας την βάση αυτή.\\
\pSpaceΜια επέκταση αυτής της εφαρμογής είναι να προσθετεί η δυνατότητα προβολής \e{Agile} των εργασιών.Παρόλο, που αυτή η λειτουργία ήταν διαθέσιμη, δεν αποτελούσε μια πρακτική λύση στην μορφή που βρισκόταν.\\
\pSpaceΜια άλλη ιδέα επέκτασης είναι η ενσωμάτωση διάφορων γνωστών εργαλείων, όπως \e{Dropbox, GoogleDrive, Slack} η \e{Github}.Αυτό θα επηρεάζει σημαντικά την αποτελεσματικότητα διαχείρισης ενός έργου.\\
\pSpaceΗ εφαρμογή μπορεί να βελτιωθεί ξεκινώντας από τις υπηρεσίες ειδοποιήσεων και προσκλήσεων, έτσι ώστε η εμπειρία του χρήστη να βελτιωθεί επίσης.Προς το παρόν οι δύο αναφερόμενες υπηρεσίες στερούν στό επίπεδο χρησιμότητας που επιθυμείται.Το ίδιο ισχύει για την εφαρμογή σε κινητό περιβάλλον, όπου ίσως το σχεδιασμό να είναι το μεγαλύτερο πρόβλημα.