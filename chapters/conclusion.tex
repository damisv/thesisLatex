\subsection*{Συμπεράσματα}
\pSpace Οι εφαρμογές και το διαδίκτυο αποτελούν σήμερα τα απαραίτητα εργαλεία στην διαχείριση έργων. Καθώς η τεχνολογία προχωράει και οι θεωρίες πολλαλασιάζονται περί του θέματος, οι χρήστες ψάχνουν την καλύτερη δυνατή λύση. Όμως, ο συνδυασμός των βασικών αναγκών, η εύκολη χρήση και η εξοικονόμηση χρόνου οδηγεί τους δημιουργούς να ψάχνουν ανάμεσα απο τις καλύτερες λύσεις, που προσφέρονται στην αγορά, μια λύση που να εκπληρώνει τον συνδυασμό αυτό.\\
\pSpace Στην αρχή της παρούσας εργασίας ο σκοπός που τέθηκε ήταν η κατασκεύη μια διαδικτυακής εφαρμογής για την διαχείριση έργων, που στοχεύει την απλότητα και εύκολη χρήση της.\\
\pSpace Ύστερα από κάποιες δοκιμές απο άτομα με διαφορετικό επίπεδο γνώσεων ως πρός το \e{Project Management}, φτάσαμε στο συμπεράσμα πως η εφαρμογή έχει εκπληρώσει τον σκοπό της. Ενώ στερεί σε αριθμό λειτουργιών, παρατηρήθηκε πως ο χρήστης ήταν ικανοποιημένος κυρίως από την απλή χρήση και την εξοικονόμηση χρόνου. Όπως ήταν αναμενόμενο, στις περιπτώσεις όπου τα έργα ήταν μεγαλύτερου μεγέθους, απαιτείται μια λύση που να καλύπτει περισσότερες λειτουργίες.\\
\pSpace Ένας δεύτερος σκοπός που έχει τεθεί, ως προς την υλοποίηση, ήταν ο σωστός συνδυασμός των δύο γνωστών πρότυπων προγραμματισμού, Αντικειμενοστραφής και Αντιδραστικός.\\
\pSpace Το δυσκολότερο κομμάτι για την επίτευξη αυτού του σκοπού, ήταν η συνήθεια στην Αντιδραστική σκέψη. Χρησιμοποιώντας αύτο το πρότυπο, άλλες πρακτικές του προγραμματισμού έπρεπε να απορριφθούν καθώς ήταν ανούσιες η έκαναν τον κώδικα πολύπλοκο χωρίς κάποιο κέρδος.\\
\pSpace Φτάσαμε στο συμπεράσμα όμως, ότι ο συνδυασμός των δύο δεν συνιστάται αν οι προγραμματιστές δεν αποφασίσουν ομαδικά να ακολουθούν ένα αντιδραστικό τρόπο ανάπτυξης. Επίσης, η επίτευξη αυτού του σκοπού θεωρήθηκε ομαδικώς, ότι έχει ολοκληρωθεί κατά 85 $\%$.\\

\subsection*{Επεκτασιμότητα και βελτιώσεις}
\pSpace Η εφαρμογή βρίσκεται αύτην την στιγμή σε ένα πολύ βασικό στάδιο, αλλά μπορεί να αναπτυχθεί σε αρκετά υψηλότερο επίπεδο κρατώντας την ίδια βάση.\\
\pSpace Μια επέκταση αυτής της εφαρμογής είναι να προστεθεί η δυνατότητα προβολής \e{Agile} των εργασιών. Παρόλο, που αυτή η λειτουργία ήταν διαθέσιμη, δεν αποτελούσε μια πρακτική λύση στην μορφή που βρισκόταν.\\
\pSpace Μια άλλη ιδέα επέκτασης είναι η ενσωμάτωση διάφορων γνωστών εργαλείων, όπως \e{Dropbox, GoogleDrive, Slack} η \e{Github}.\\
\pSpace Η εφαρμογή μπορεί να βελτιωθεί ξεκινώντας από τις υπηρεσίες ειδοποιήσεων και προσκλήσεων, έτσι ώστε η εμπειρία του χρήστη να βελτιωθεί. Προς το παρόν οι δύο αναφερόμενες υπηρεσίες στερούν στό επίπεδο χρησιμότητας που επιθυμείται. Το ίδιο ισχύει για την εφαρμογή σε κινητό περιβάλλον, όπου ίσως ο σχεδιασμός να είναι το μεγαλύτερο πρόβλημα.