\pSpaceΟ Παγκόσμιος Ιστός αποτελεί σήμερα το κύριο μέσο επικοινωνίας, μάθησης και ανάπτυξης λόγω του τεράστιου χώρο διακίνησης όγκου πληροφοριών.Πλέον ο ρόλος του στην ζωή των ανθρώπον είναι καθοριστικό και αναγκαίο.Επομένως γεννήθηκαν ιδέες και αναπτύχθηκαν εκκατομμύρια εφαρμογές για την απλοποίηση υποχρεώσεων των χρηστών παγκοσμίος, απ'ο την στιγμή που εμφανίστηκε στην πρώτη του μορφή το 1989.Μια από αυτές τις ιδέες είναι το \e{Project Management}.\\
\pSpaceΗ Διοίκηση και Διαχείριση έργων (\e{Project Management}) σύμφωνα με τον ορισμός που προσφέρει Βικιπαιδεία, είναι η πρακτική της έναρξης, προγραμματισμού, εκτέλεσης, ελέγχου και κλεισίματος έργου μιας ομάδας για την επίτευξη συγκεκριμένων στόχων και την εκπλήρωση συγκεκριμένων κριτήριων επιτυχίας σε καθορισμένο χρόνο.\\
\pSpaceΗ σημερινή μορφή του \e{project management} έχει τις αρχές στην δεκαετία του 1950, αλλά ο πατέρας του γνωστικού πεδίου της διαχείρισης έργων θεωρείται ο Χένρι Γκαντ (\e{Henry Gantt 1861 - 1919}) ο οποίος έθεσε τις βάσεις του προγραμματισμού και ελέγχου στην διαχείριση ενός έργου.\\
\pSpaceΕφαρμόζοντας λοιπόν την θεωρία Διοίκισης και Διαχείρισης έργων, η ανάπτυξη και η επικοινωνία μιας ομάδας διευκολύνεται σημαντικά ενώ ο χρόνος παράδοσης τελικού εποτελέσματος ελαχιστοποιείται.Για τον λόγο αυτό ο χώρος της διαχείρισης έρων προσελκύει ιδιαίτερο ενδιαφέρον στον ιδιωτικό και δημόσιο τομέα οπώς και στην ακαδημαική ταυτότητα.

\subsection*{Σκοπός της πτυχιακής}
\pSpaceΤο βασικό ζήτημα μιας σωστής διοίκησης και διαχείρισης ενός έργου, είναι ο χώρος όπου μπορούν να συγκεντρωθούν όλες οι εργασίες και θέματα περί του έργου έτσι ώστε να γίνουν εμφανείς στον κάθε μέλος μιας ομάδας ποια είναι τα βήματα που πρέπει να ακολουθήσει.\\
\pSpaceΩς σκοπός της πτυχιακής, τέθηκε η κατασκευή μιας εφαρμογής για την διαχείριση έργων που στοχεύει την απλότητα και εύκολη χρήσης της, έτσι ώστε να μπορεί να χρησιμοποιηθεί σε διάφορους τομείς ανεξαρτήτος γνώσεων μελών.

\subsection*{Οργάνωση του τόμου}
\pSpaceΗ εργασία αυτή είναι οργανωμένη σε 6 κεφάλαια.
\begin{itemize}
    \item Στο Κεφάλαιο 2 δίνεται το υπόβαθρο των τεχνολογιών που χρησημοποιήθηκαν στην ανάπτυξη της εφαρμογής.
    \item Στο Κεφάλαιο 3 βρίσκονται οι απαιτήσεις προδιαγραφών, τεχνική περιγραφή γενικής όσο και ειδικής ανά λειτουργία και μια έρευνα αγοράς για τις σύχρονες εφαρμογές ίδιου σκοπού.
    \item Το Κεφάλαιο 4 παρουσιάζει τα βήματα για το \e{deploy} της εφαρμογής στο \e{cloud}.
    \item Στο Κεφάλαιο 5 υπάρχει ο εγχειρίδιο χρήσης ως προς κάθε λειτουργία της εφαρμογής.
    \item Και τέλος, στο Κεφάλαιο 6 διατυπώνονται τα συμπεράσματα που αφορούν τους αρχικούς στόχους που τέθηκαν, αλλά και μελλοντικές πιθανές εξελίξεις της εφαρμογής.
\end{itemize}